\documentclass[conference]{IEEEtran}
\IEEEoverridecommandlockouts
% The preceding line is only needed to identify funding in the first footnote. If that is unneeded, please comment it out.
\usepackage{cite}
\usepackage{amsmath,amssymb,amsfonts}
\usepackage{algorithmic}
\usepackage{graphicx}
\usepackage{textcomp}
\usepackage{xcolor}
\def\BibTeX{{\rm B\kern-.05em{\sc i\kern-.025em b}\kern-.08em
    T\kern-.1667em\lower.7ex\hbox{E}\kern-.125emX}}
\begin{document}

\title{A Strategy Video-Game for Collaborative Agents with a Personality and Humoristic Dialogues\\
{\large Context Survey}
}

\author{\IEEEauthorblockN{Jordan Mackie}
\IEEEauthorblockA{\textit{Student}}
\and
\IEEEauthorblockN{Alice Toniolo}
\IEEEauthorblockA{\textit{Supervisor}}
\and
\IEEEauthorblockN{Christopher Stone}
\IEEEauthorblockA{\textit{Supervisor}}
}

\maketitle

\begin{abstract}

	In this document, we provide a background for the areas of research related to the project, and some discussion of the tools and technologies that will be used. After describing the goals and results of previous research, we identify the components that could be reused or built upon for the purposes of building a system of collaborative agents with personality driven decisions and humoristic dialogues, and why the results of this project could be useful for areas such as human-computer interaction, modelling, and entertainment. 

\end{abstract}

\section{Introduction}

Multiagent systems are the next step to increase the level of autonomy that can be provided by technology. Agents that are able to learn, adapt, and negotiate with other agents to achieve their goals allow for complex problems to be solved or modelled, such as monitoring and maintaining national power grids \cite{archon}. 

Often, developers and users will anthropomorphise these agents when describing their behaviour. This project aims to encourage this by implementing agents with a model of personality that affects their choice of actions, by rendering the negotiations between agents in a natural language, and by using models of humour to make the interactions between agents entertaining.

\section{Survey}

\subsection{Multiagent Systems}

\subsection{Models of Personality}
\cite{maven}

\subsection{Computational Humour and Natural Language Generation}

\begin{thebibliography}{00}
		\bibitem asdf
\end{thebibliography}
\vspace{12pt}

\end{document}
