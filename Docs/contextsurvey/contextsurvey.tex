\documentclass[conference]{IEEEtran}
\IEEEoverridecommandlockouts
% The preceding line is only needed to identify funding in the first footnote. If that is unneeded, please comment it out.
\usepackage{cite}
\usepackage{amsmath,amssymb,amsfonts}
\usepackage{algorithmic}
\usepackage{graphicx}
\usepackage{textcomp}
\usepackage{xcolor}
\def\BibTeX{{\rm B\kern-.05em{\sc i\kern-.025em b}\kern-.08em
    T\kern-.1667em\lower.7ex\hbox{E}\kern-.125emX}}
\begin{document}

\title{A Strategy Video-Game for Collaborative Agents with a Personality and Humoristic Dialogues\\
{\large Context Survey}
}

\author{\IEEEauthorblockN{Jordan Mackie}
\IEEEauthorblockA{\textit{Student}}
\and
\IEEEauthorblockN{Alice Toniolo}
\IEEEauthorblockA{\textit{Supervisor}}
\and
\IEEEauthorblockN{Christopher Stone}
\IEEEauthorblockA{\textit{Supervisor}}
}

\maketitle

\begin{abstract}

	In this document, we provide a background for the areas of research related to the project, and some discussion of the tools and technologies that will be used. After describing the goals and results of previous research, we identify the components that could be reused or built upon for the purposes of building a system of collaborative agents with personality driven decisions and humouristic dialogues, and why the results of this project could be useful for areas such as human-computer interaction, modelling, and entertainment. 

\end{abstract}

\begin{IEEEkeywords}
multiagent, personality, humour, strategy
\end{IEEEkeywords}

\section{Introduction}

	Overview of what multiagent systems are

\section{Survey}

\subsection{Multiagent Systems}

\subsection{Models of Personality}
\cite{maven}

\subsection{Computational Humour and Natural Language Generation}

\begin{thebibliography}{00}
\bibitem{b1} G. Eason, B. Noble, and I. N. Sneddon, ``On certain integrals of Lipschitz-Hankel type involving products of Bessel functions,'' Phil. Trans. Roy. Soc. London, vol. A247, pp. 529--551, April 1955.
\bibitem{b2} J. Clerk Maxwell, A Treatise on Electricity and Magnetism, 3rd ed., vol. 2. Oxford: Clarendon, 1892, pp.68--73.
\bibitem{b3} I. S. Jacobs and C. P. Bean, ``Fine particles, thin films and exchange anisotropy,'' in Magnetism, vol. III, G. T. Rado and H. Suhl, Eds. New York: Academic, 1963, pp. 271--350.
\bibitem{b4} K. Elissa, ``Title of paper if known,'' unpublished.
\bibitem{b5} R. Nicole, ``Title of paper with only first word capitalized,'' J. Name Stand. Abbrev., in press.
\bibitem{b6} Y. Yorozu, M. Hirano, K. Oka, and Y. Tagawa, ``Electron spectroscopy studies on magneto-optical media and plastic substrate interface,'' IEEE Transl. J. Magn. Japan, vol. 2, pp. 740--741, August 1987 [Digests 9th Annual Conf. Magnetics Japan, p. 301, 1982].
\bibitem{b7} M. Young, The Technical Writer's Handbook. Mill Valley, CA: University Science, 1989.
\end{thebibliography}
\vspace{12pt}

\end{document}
