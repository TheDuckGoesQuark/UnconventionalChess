\documentclass[conference]{IEEEtran}
\IEEEoverridecommandlockouts
% The preceding line is only needed to identify funding in the first footnote. If that is unneeded, please comment it out.
\usepackage{cite}
\usepackage{amsmath,amssymb,amsfonts}
\usepackage{algorithmic}
\usepackage{graphicx}
\usepackage{textcomp}
\usepackage{xcolor}
\def\BibTeX{{\rm B\kern-.05em{\sc i\kern-.025em b}\kern-.08em
    T\kern-.1667em\lower.7ex\hbox{E}\kern-.125emX}}
\begin{document}

\title{A Strategy Video-Game for Collaborative Agents with a Personality and Humoristic Dialogues\\
{\large Context Survey}
}

\author{\IEEEauthorblockN{Jordan Mackie}
\IEEEauthorblockA{\textit{Student}}
\and
\IEEEauthorblockN{Alice Toniolo}
\IEEEauthorblockA{\textit{Supervisor}}
\and
\IEEEauthorblockN{Christopher Stone}
\IEEEauthorblockA{\textit{Supervisor}}
}

\maketitle

\begin{abstract}

	In this document, we provide a background for the areas of research related to the project, and some discussion of the tools and technologies that will be used. After describing the goals and results of previous research, we identify the components that could be reused or built upon for the purposes of building a system of collaborative agents with personality driven decisions and humoristic dialogues, and why the results of this project could be useful for areas such as human-computer interaction, modelling, and entertainment. 

\end{abstract}

\section{Introduction}

Multiagent systems are the next step to increase the level of autonomy that can be provided by technology. Agents that are able to learn, adapt, and negotiate with other agents to achieve their goals allow for complex problems to be solved or modelled, such as monitoring and maintaining national power grids \cite{archon}. 

Often, developers and users will anthropomorphise these agents when describing their behaviour. This project aims to encourage this by implementing agents with a model of personality that affects their choice of actions, by rendering the negotiations between agents in a natural language, and by using models of humour to make the interactions between agents entertaining.

\section{Survey}

\subsection{Multiagent Systems in Entertainment}

Many modern video games involve the user managing multiple characters to achieve some goal, such as producing in-game resources or defending against an opponent. This sort of problem lends itself easily to multiagent systems. Instead of having one artificial intelligence engine driving the actions of all the characters, developers can create agents with a limited set of actions and some concept of progress towards their goals and allow emergent behaviour find a solution to the problem, sometimes in surprising ways. 

\cite{sandboxmas} discuss how multiagent systems can be applied to create more realistic worlds in sandbox games. By implementing the environment, objects, and non-playable characters (NPCs) as agents, developers can create a world that reacts and adapts the player, but also operates in isolation from the character to provide a realistic setting. The concept of personalities is also mentioned as a way of allowing similar NPCs to exhibit different slightly different behaviours, such as having aggressive or relaxed driving styles.

Even in games with very simple rules and logic, multiagent systems can find interesting and complex solutions. OpenAI implemented hide-and-seek using agents, where hiders avoid the line-of-sight of the seekers \cite{openaiemergent}. The game was played in a world with randomly generated walls and objects such as ramps (for climbing over walls) and blocks (for forming barricades). What made this fascinating is that the agents were not incentivised to use these objects, but after repeatedly playing and learning, both the hiders and seekers created strategies such as blocking each other in a safe area, and even removing the objects from the other team before hiding.

Knowing that multi-agent autocurricula can realise strategies not considered by humans, \cite{masboardgames} discuss how they can be applied to strategy board games such as Diplomacy and Risk. They describe a generic framework for supporting agent-based competitive bots for board games and then implement bots for the aforementioned games. Their research suggests that for games with a large number of units and a large action space, a multi-agent approach can identify effective strategies quicker than the exhaustive methods used for chess engines. 

\subsection{Models of Personality and Emotion}

Creating a truly immersive video game requires characters that the player can empathise with. Robots have been shown to be able to influence human behaviour as an authority figure \cite{bossrobot} and when begging not to be turned off \cite{turnoffrobot} by expressing emotions. \cite{personalitymodel} created and demonstrated a model of personality and emotion that would allow agents to react differently to the same stimulation. For example, when a agent with an 'introverted' personality is offered help, they are less likely to accept it due to the prolonged interaction it would entail. 

	\cite{skyrim} also created a model that would use personality, emotion, and social relationships to determine the behaviour of NPCs in a video game. The frequency and tone of interactions between NPCs as well as the NPC and the player were accounted for when choosing facial expressions and tone of voice during conversations. Test subjects described feeling especially attached to the NPCs that utilised this model showing.

A useful aspect of multiagent systems is that agents can be developed in isolation but still interact (e.g. an agent that searches for cheap transport options and an agent responsible for auctioning train tickets could be developed seperately with no knowledge of the logic being used by the other). \cite{hetrogenousagents} discuss how developing heterogenous agent systems using personalities and social structures could help when dealing with third-party agents that have been constructed to lie and exhibit selfish or uncooperative behaviour.

\subsection{Computational Humour and Natural Language Generation}



\begin{thebibliography}{00}
		\bibitem asdf
\end{thebibliography}
\vspace{12pt}

\end{document}
