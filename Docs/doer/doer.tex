\documentclass[12pt]{extarticle}
\usepackage[utf8]{inputenc}
\usepackage{cite}


\title{%
  A Strategy Video-Game for Collaborative Agents with a Personality and Humoristic Dialogues \\
  \vspace{5mm} 
  \large Description, Objectives, Ethics, Resources \\
  (DOER)}

\author{\textit{Author}: Jordan Mackie \and 
{\textit{Supervisors}: Alice Toniolo, Christopher Stone}}

\date{\today}

\begin{document}

\maketitle

\pagenumbering{gobble}

\section*{Description}

This project involves creating a strategy video-game for collaborative agents. Agent behaviour will be partially determined by some implementation of a personality, and interactions between agents will be translated to human-readable dialogue. The main area of interest in this project is that the behaviour of and dialogue between agents will be constructed to be entertaining and humorous for either the human adversary or observer. Chess will be used as the strategy game, and the pieces will each act as individual agents.

In a typical multiagent system, agents cooperate to achieve a common goal through a series of decisions. The actions taken by agents are chosen due to being the next most rational step to a solution. This project intends to introduce factors such as boredom, excitement and other human-like features into the decision making process of agents. By using natural language generation techniques, the discussions between agents will be displayed to the user so that they can follow along and understand the choices of agents.

The humoristic behaviour and dialogue between agents will require investigation into computational models of humour. Agents may cooperate to produce entertaining scenarios, or produce them independently by executing actions outside of whatever consensus is reached between agents.

The project has obvious benefits for game development: introducing humour and personalities to characters in any game increases immersion and improves the overall experience for the user. These concepts could also be applied to the world of robotics and digital assistants in order to make the interaction between users and machine more natural.

\section*{Objectives}

The main artefacts of this project are:
\begin{enumerate}
	\item Design and implementation of the game infrastructure
	\item Design and implementation of a decision-making model for agents with personalities that produces entertaining and humorous behaviour.
	\item Design and implementation of human understandable dialogue constructed from agent interactions and intentions.
\end{enumerate}

\section*{Ethics}

Here you should discuss any ethical considerations
pertaining to your project. Start with the self-
assessment form from the Student Handbook (Ethics
section). If you can answer “No” to all questions on the
self-assessment form, this section of the DOER
document will be brief and state that there are no ethical
considerations.
If you are planning to work with people (especially
children), animals, sensitive private data, or if there are
other considerations, you should discuss them here, and
explain how you went about obtaining necessary
approval (any Ethics applications).
The self-assessment form and any other relevant
documents (if applicable) should be scanned and
uploaded to the “Ethics” slot on MMS.

\section*{Resources}

This is a list of any special resources your project will
need: hardware, software, licenses, access to
infrastructure (e.g. compute servers), drones, etc. Think
ahead, but be realistic -- the School will not be able to
fulfill all requests.
Most projects can be completed using standard school
equipment, in which case this section will contain only a
short statement confirming this.

\end{document}
