\documentclass[12pt]{extarticle}

\usepackage[utf8]{inputenc}
\usepackage{cite}
\newcommand\descitem[1]{\item{\bfseries #1}\\}

\title{%
  A Strategy Video-Game for Collaborative Agents with a Personality and Humoristic Dialogues \\
  \vspace{5mm} 
  \large Description, Objectives, Ethics, Resources \\
  (DOER)}

\author{\textit{Author}: Jordan Mackie \and 
{\textit{Supervisors}: Alice Toniolo, Christopher Stone}}

\date{\today}

\begin{document}

\maketitle

\pagenumbering{gobble}

\section*{Description}

This project involves creating a strategy video-game for collaborative agents. Agent behaviour will be partially determined by some implementation of a personality, and interactions between agents will be translated to human-readable dialogue. The main area of interest in this project is that the behaviour of and dialogue between agents will be constructed to be entertaining and humorous for either the human adversary or observer. Chess will be used as the strategy game, and the pieces will each act as individual agents.

In a typical multiagent system, agents cooperate to achieve a common goal through a series of decisions. The actions taken by agents are chosen due to being the next most rational step to a solution. This project intends to introduce factors such as boredom, excitement and other human-like features into the decision making process of agents. By using natural language generation techniques, the discussions between agents will be displayed to the user so that they can follow along and understand the choices of agents.

The humoristic behaviour and dialogue between agents will require investigation into computational models of humour. Agents may cooperate to produce entertaining scenarios, or produce them independently by executing actions outside of whatever consensus is reached between agents.

The project has obvious benefits for game development: introducing humour and personalities to characters in any game increases immersion and improves the overall experience for the user. These concepts could also be applied to the world of robotics and digital assistants in order to make the interaction between users and machine more natural.

\section*{Objectives}

\subsection*{Primary}
\begin{enumerate}
		\descitem{Base infrastructure composed of agents representing pieces, which communicate between each other to choose the next move.}
		The agents are capable of identifying optional moves at a point in the game, reaching a consensus on what piece is to move next, and each move reaches some form of textual interface.

		\descitem{Agent decisions are driven by some form of 'personality' model.}
		Possibly a set of static weightings that push them to have certain desires and be more likely to propose certain actions over others.

		\descitem{Dialogue is generated related to the discussions between agents.}
		Human-readable dialogue is generated to describe the discussions taking place between agents.

		\descitem{Humour is achieved through either the human-readable dialogue, or from the actions of agents.}
		A fixed set of rules inspired by computational models of humour that define how a character should behave in consecutive dialogues. For example, agents break from 'expected' behaviour to produce funny situations, or conversations between agents contain satirical comments related to the current state of the game.

\end{enumerate}

\subsection*{Secondary}

\begin{enumerate}
		\descitem{Personalities and behaviours change over time in reaction to the actions of other agents or the current state of the game}
		The weightings that drive individual agent decisions are affected by factors such as how cooperative other agents are being, or how often this piece is threatened.

		\descitem{Dialogue continues throughout game, not just during the main decision making process.}
		The agents will constantly produce dialogue even when not making decisions. This could be in reaction to the move just made, or be directed at the player.

		\descitem{Thorough evaluation by human users}
		The game is tested by a number of users, and feedback is collected as to how well the game achieves its goal of being entertaining. This feedback is used to describe possible next steps for the project.

\end{enumerate}

\subsection*{Tertiary}

\begin{enumerate}

		\descitem{The 'hierarchical' nature of chess pieces is accounted for during the decision making process between agents}
		For example, actions suggested by bishops and knights will be more likely to be accepted compared to that of a pawn. This hierarchy could also be dynamic, and make agents in more powerful positions (for example, a rook cornering the enemy king) also have more weighting during the decision making process.

		\descitem{Detailed Graphical Interface}
		The game has high quality visuals that reflect the state of the game, animations, and other aesthetic properties that make it more enjoyable to play.

\end{enumerate}

\section*{Ethics}

Some short feedback tests will be run with volunteers to evaluate the success of the project. Because of this, the preliminary self-assessment form and the artifact evaluation form will also be submitted to cover these limited tests.

\section*{Resources}

This project will only require lab machines for development and testing purposes.

\end{document}

