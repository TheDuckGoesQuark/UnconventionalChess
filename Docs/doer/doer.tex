\documentclass[12pt]{extarticle}

\usepackage[utf8]{inputenc}
\usepackage{cite}
\newcommand\descitem[1]{\item{\bfseries #1}\\}

\title{%
  A Strategy Video-Game for Collaborative Agents with a Personality and Humoristic Dialogues \\
  \vspace{5mm} 
  \large Description, Objectives, Ethics, Resources \\
  (DOER)}

\author{\textit{Author}: Jordan Mackie \and 
{\textit{Supervisors}: Alice Toniolo, Christopher Stone}}

\date{\today}

\begin{document}

\maketitle

\pagenumbering{gobble}

\section*{Description}

This project involves creating a strategy video-game for collaborative agents. Agent behaviour will be partially determined by some implementation of a personality, and interactions between agents will be translated to human-readable dialogue. The main area of interest in this project is that the behaviour of and dialogue between agents will be constructed to be entertaining and humorous for either the human adversary or observer. Chess will be used as the strategy game, and the pieces will each act as individual agents.

In a typical multiagent system, agents cooperate to achieve a common goal through a series of decisions. The actions taken by agents are chosen due to being the next most rational step to a solution. This project intends to introduce factors such as boredom, excitement and other human-like features into the decision making process of agents. By using natural language generation techniques, the discussions between agents will be displayed to the user so that they can follow along and understand the choices of agents.

The humoristic behaviour and dialogue between agents will require investigation into computational models of humour. Agents may cooperate to produce entertaining scenarios, or produce them independently by executing actions outside of whatever consensus is reached between agents.

The project has obvious benefits for game development: introducing humour and personalities to characters in any game increases immersion and improves the overall experience for the user. These concepts could also be applied to the world of robotics and digital assistants in order to make the interaction between users and machine more natural.

\section*{Objectives}

\begin{enumerate}
		\descitem{Design and implementation of the game infrastructure} 
		Easy to use and well tested, with the source having a clear and logical structure. 

		\descitem{Design and implementation of a decision-making model for agents with personalities that produces humorous behaviour.} 
		Altering initial conditions (e.g. personality traits of pieces) should produce novel and entertaining results, whilst still providing some challenge to the user. The focus is not on creating a winning AI, but a level of skill is expected.

		\descitem {Design and implementation of human understandable dialogue constructed from agent interactions and intentions.} 
		A human player should be able to follow the interactions between pieces in the form of textual dialogue and understand why its opposition is making certain moves. 

\end{enumerate}

\section*{Ethics}

This project does not require any ethical considerations. Some human testers may be employed to determine the success of the project, and if so the necessary procedures for approval will be taken.

\section*{Resources}

This project will only require lab machines for development and testing purposes.

\end{document}
