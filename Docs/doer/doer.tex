%%%%%%%%%%%%%%%%%%%%%%%%%%%%%%%%%%%%%%%%%
% University Assignment Title Page 
% LaTeX Template
% Version 1.0 (27/12/12)
%
% This template has been downloaded from:
% http://www.LaTeXTemplates.com
%
% Original author:
% WikiBooks (http://en.wikibooks.org/wiki/LaTeX/Title_Creation)
%
% License:
% CC BY-NC-SA 3.0 (http://creativecommons.org/licenses/by-nc-sa/3.0/)
%
%%%%%%%%%%%%%%%%%%%%%%%%%%%%%%%%%%%%%%%%%
%\title{Title page with logo}
%----------------------------------------------------------------------------------------
%	PACKAGES AND OTHER DOCUMENT CONFIGURATIONS
%----------------------------------------------------------------------------------------

\documentclass[12pt]{article}
\usepackage[english]{babel}
\usepackage[utf8]{inputenc}
\usepackage{color}

\begin{document}

\section{Description}

The title and a short description of the project aims,
context and background. It should explain the big
picture of what you would like to achieve, why it is
important, and how you intend to go about doing it (e.g.
by using some kind of technology or developing a new
algorithm, or following a particular methodology, etc.)

\section{Objectives}

This is a list of clearly defined, measurable goals you
intend to achieve by the end of your project. This could
include any software artefacts you intend to submit in
the end, results of an evaluation (for surveys or research
algorithms), etc. Your performance will be measured
against these objectives.
Typically, you will list about 3-5 primary objectives
which are necessary for a project to be deemed
successful, and further 3 or so secondary objectives
which allow a successful project to be extended in an
interesting direction. Occasionally, tertiary objectives
may also be listed, but these are comparatively rare.

\section{Ethics}

Here you should discuss any ethical considerations
pertaining to your project. Start with the self-
assessment form from the Student Handbook (Ethics
section). If you can answer “No” to all questions on the
self-assessment form, this section of the DOER
document will be brief and state that there are no ethical
considerations.
If you are planning to work with people (especially
children), animals, sensitive private data, or if there are
other considerations, you should discuss them here, and
explain how you went about obtaining necessary
approval (any Ethics applications).
The self-assessment form and any other relevant
documents (if applicable) should be scanned and
uploaded to the “Ethics” slot on MMS.

\section{Resources}

This is a list of any special resources your project will
need: hardware, software, licenses, access to
infrastructure (e.g. compute servers), drones, etc. Think
ahead, but be realistic -- the School will not be able to
fulfill all requests.
Most projects can be completed using standard school
equipment, in which case this section will contain only a
short statement confirming this.

\end{document}
